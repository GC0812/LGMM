\documentclass[12pt]{article}
%%---------------------------------------------------------------------
% packages
% geometry
\usepackage{geometry}
% font
\usepackage{fontspec}
\defaultfontfeatures{Mapping=tex-text}  %%如果没有它,会有一些 tex 特殊字符无法正常使用,比如连字符。
\usepackage{xunicode,xltxtra}
\usepackage[BoldFont,SlantFont,CJKnumber,CJKchecksingle]{xeCJK}  % \CJKnumber{12345}: 一万二千三百四十五
\usepackage{CJKfntef}  %%实现对汉字加点、下划线等。
\usepackage{pifont}  % \ding{}
% math
\usepackage{amsmath,amsfonts,amssymb}
% color
\usepackage{color}
\usepackage{xcolor}
\definecolor{EYE}{RGB}{199,237,204}
\definecolor{FLY}{RGB}{128,0,128}
\definecolor{ZHY}{RGB}{139,0,255}
% graphics
\usepackage[americaninductors,europeanresistors]{circuitikz}
\usepackage{tikz}
\usetikzlibrary{positioning,arrows,shadows,shapes,calc,mindmap,trees,backgrounds}  % placements=positioning
\usepackage{graphicx}  % \includegraphics[]{}
\usepackage{subfigure}  %%图形或表格并排排列
% table
\usepackage{colortbl,dcolumn}  %% 彩色表格
\usepackage{multirow}
\usepackage{multicol}
\usepackage{booktabs}
% code
\usepackage{fancyvrb}
\usepackage{listings}
% title
\usepackage{titlesec}
% head/foot
\usepackage{fancyhdr}
% ref
\usepackage{hyperref}
% pagecolor
\usepackage[pagecolor={EYE}]{pagecolor}
% tightly-packed lists
\usepackage{mdwlist}
%\usepackage[noend]{algpseudocode}
\usepackage{algorithm}
\usepackage{algorithmic}

\usepackage{styles/iplouccfg}
\usepackage{styles/zhfontcfg}
\usepackage{styles/iplouclistings}

%%---------------------------------------------------------------------
% settings
% geometry
\geometry{left=2cm,right=1cm,top=2cm,bottom=2cm}  %设置 上、左、下、右 页边距
\linespread{1.5} %行间距
% font
\setCJKmainfont{Adobe Kaiti Std}
%\setmainfont[BoldFont=Adobe Garamond Pro Bold]{Apple Garamond}  % 英文字体
%\setmainfont[BoldFont=Adobe Garamond Pro Bold,SmallCapsFont=Apple Garamond,SmallCapsFeatures={Scale=0.7}]{Apple Garamond}  %%苹果字体没有SmallCaps
\setCJKmonofont{Adobe Fangsong Std}
% graphics
\graphicspath{{figures/}}
\tikzset{
    % Define standard arrow tip
    >=stealth',
    % Define style for boxes
    punkt/.style={
           rectangle,
           rounded corners,
           draw=black, very thick,
           text width=6.5em,
           minimum height=2em,
           text centered},
    % Define arrow style
    pil/.style={
           ->,
           thick,
           shorten <=2pt,
           shorten >=2pt,},
    % Define style for FlyZhyBall
    FlyZhyBall/.style={
      circle,
      minimum size=6mm,
      inner sep=0.5pt,
      ball color=red!50!blue,
      text=white,},
    % Define style for FlyZhyRectangle
    FlyZhyRectangle/.style={
      rectangle,
      rounded corners,
      minimum size=6mm,
      ball color=red!50!blue,
      text=white,},
    % Define style for zhyfly
    zhyfly/.style={
      rectangle,
      rounded corners,
      minimum size=6mm,
      ball color=red!25!blue,
      text=white,},
    % Define style for new rectangle
    nrectangle/.style={
      rectangle,
      draw=#1!50,
      fill=#1!20,
      minimum size=5mm,
      inner sep=0.1pt,}
}
\ctikzset{
  bipoles/length=.8cm
}
% code
\lstnewenvironment{VHDLcode}[1][]{%
  \lstset{
    basicstyle=\footnotesize\ttfamily\color{black},%
    columns=flexible,%
    framexleftmargin=.7mm,frame=shadowbox,%
    rulesepcolor=\color{blue},%
%    frame=single,%
    backgroundcolor=\color{yellow!20},%
    xleftmargin=1.2\fboxsep,%
    xrightmargin=.7\fboxsep,%
    numbers=left,numberstyle=\tiny\color{blue},%
    numberblanklines=false,numbersep=7pt,%
    language=VHDL%
    }\lstset{#1}}{}
\lstnewenvironment{VHDLmiddle}[1][]{%
  \lstset{
    basicstyle=\scriptsize\ttfamily\color{black},%
    columns=flexible,%
    framexleftmargin=.7mm,frame=shadowbox,%
    rulesepcolor=\color{blue},%
%    frame=single,%
    backgroundcolor=\color{yellow!20},%
    xleftmargin=1.2\fboxsep,%
    xrightmargin=.7\fboxsep,%
    numbers=left,numberstyle=\tiny\color{blue},%
    numberblanklines=false,numbersep=7pt,%
    language=VHDL%
    }\lstset{#1}}{}
\lstnewenvironment{VHDLsmall}[1][]{%
  \lstset{
    basicstyle=\tiny\ttfamily\color{black},%
    columns=flexible,%
    framexleftmargin=.7mm,frame=shadowbox,%
    rulesepcolor=\color{blue},%
%    frame=single,%
    backgroundcolor=\color{yellow!20},%
    xleftmargin=1.2\fboxsep,%
    xrightmargin=.7\fboxsep,%
    numbers=left,numberstyle=\tiny\color{blue},%
    numberblanklines=false,numbersep=7pt,%
    language=VHDL%
    }\lstset{#1}}{}
% pdf
\hypersetup{pdfpagemode=FullScreen,%
            pdfauthor={Haiyong Zheng},%
            pdftitle={Title},%
            CJKbookmarks=true,%
            bookmarksnumbered=true,%
            bookmarksopen=false,%
            plainpages=false,%
            colorlinks=true,%
            citecolor=green,%
            filecolor=magenta,%
            linkcolor=cyan,%red(default)
            urlcolor=cyan}
% section
%http://tex.stackexchange.com/questions/34288/how-to-place-a-shaded-box-around-a-section-label-and-name
\newcommand\titlebar{%
\tikz[baseline,trim left=3.1cm,trim right=3cm] {
    \fill [cyan!25] (2.5cm,-1ex) rectangle (\textwidth+3.1cm,2.5ex);
    \node [
        fill=cyan!60!white,
        anchor= base east,
        rounded rectangle,
        minimum height=3.5ex] at (3cm,0) {
        \textbf{\thesection.}
    };
}%
}
\titleformat{\section}{\Large\bf\color{blue}}{\titlebar}{0.1cm}{}
% head/foot
\setlength{\headheight}{15pt}
\pagestyle{fancy}
\fancyhf{}
%\lhead{\color{black!50!green}2014年秋季学期}
\chead{\color{black!50!green}}
%\rhead{\color{black!50!green}通信电子电路}
\lfoot{\color{blue!50!green}邱欣欣}
\cfoot{\color{blue!50!green}\href{http://vision.ouc.edu.cn/~zhenghaiyong}{CVBIOUC}}
\rfoot{\color{blue!50!green}$\cdot$\ \thepage\ $\cdot$}
\renewcommand{\headrulewidth}{0.4pt}
\renewcommand{\footrulewidth}{0.4pt}
\makeatletter
\def\BState{\State\hskip-\ALG@thistlm}
\makeatother
%%---------------------------------------------------------------------
\begin{document}
%%---------------------------------------------------------------------
%%---------------------------------------------------------------------
% \titlepage
\title{\vspace{-2em}Pseudocode\vspace{-0.7em}}
\author{邱欣欣}
%\date{\vspace{-0.7em}2014年11月\vspace{-0.7em}}
%%---------------------------------------------------------------------
\maketitle\thispagestyle{fancy}
%%---------------------------------------------------------------------
%\maketitle
%\tableofcontents 

%\section{Pseudocode}
\begin{algorithm}[!ht]
\caption{Space-based Depth-First Search (SDFS)}\label{alg:sdfs}
\textbf{Input}: The feature points $P$, their number $n$ and connectivity matrix $M$ in the binary skeleton image.\\
\textbf{Output}: The cycles with their feature points and connectivities in the binary skeleton image.
\begin{algorithmic}[1]
\FOR {each point $a_i \in P$}
\FOR {each point $b_j\in P$ \AND connects to $a_i$}
\STATE clear $CyclePath$ for initialization
\STATE add points $a_i$ and $b_j$ to $CyclePath$ in turn
\WHILE {$n$$-$$-$}
\STATE $p_0\leftarrow CyclePath(end-1)$
\STATE $p_1\leftarrow CyclePath(end)$
\STATE calculate vector $V_{p_1p_0}=V_{p_0}-V_{p_1}$
\FOR {each point $p_{2_k}\in P$ \AND connects to $p_1$}
\IF {$V_{p_1p_0}.x\times V_{p_1p_{2_k}}.y-V_{p_1p_0}.y\times V_{p_1p_{2_k}}.x>0$}
\STATE add $p_{2_k}$ to $RightHandPoints$
\ELSIF {$V_{p_1p_0}.x\times V_{p_1p_{2_k}}.y-V_{p_1p_0}.y\times V_{p_1p_{2_k}}.x<0$}
\STATE add $p_{2_k}$ to $LeftHandPoints$
\ELSE
\STATE add $p_{2_k}$ to $CyclePath$
\ENDIF
\ENDFOR
\IF {$RightHandPoints$ is not empty}
\FOR {each point $p_{2_m}$ in $RightHandPoints$}
\STATE calculate $\theta_{p_0p_1p_{2_m}}=\arccos{\frac{V_{p_1p_0}\cdot V_{p_1p_{2_m}}}{\left|V_{p_1p_0}\right|\left|V_{p_1p_{2_m}}\right|}}$
\ENDFOR
\STATE add $p_{2_m}$ determined by $\min{\left(\theta_{p_0p_1p_{2_m}}\right)}$ to $CyclePath$
\ELSIF {$LeftHandPoints$ is not empty}
\FOR {each point $p_{2_n}$ in $LeftHandPoints$}
\STATE calculate $\theta_{p_0p_1p_{2_n}}=\arccos{\frac{V_{p_1p_0}\cdot V_{p_1p_{2_n}}}{\left|V_{p_1p_0}\right|\left|V_{p_1p_{2_n}}\right|}}$
\ENDFOR
\STATE \textcolor{red}{add $p_{2_n}$ determined by $\min{\left(\theta_{p_0p_1p_{2_n}}\right)}$ to $CyclePath$}
\ENDIF
\IF  {$CyclePath$'s point number is greater than $2$ \AND last point is the same as first point}
\IF {\textcolor{red}{$checkCyclePath(p_0,p_1,p_2,...)$}}
\STATE output $CyclePath(p_0,p_1,p_2,...) \Rightarrow Cycles$
\STATE break
\ENDIF
\ENDIF
\ENDWHILE
\ENDFOR  
\ENDFOR
\STATE remove the duplicated $CyclePath$s from $Cycles$
\end{algorithmic}
\end{algorithm}


%%---------------------------------------------------------------------
\end{document}
